\documentclass[12pt]{article}
\usepackage{a4wide}
\usepackage{amsmath, amssymb}
\usepackage{graphicx}
\usepackage{hyperref}
\hypersetup{
    colorlinks=true,
    linkcolor=blue,
    filecolor=magenta,      
    urlcolor=blue
}

\begin{document}
\title{\vspace{-2.0cm}Effect of cold surges on extreme tides in the Western Maritime Continent}
\author{Shrinjana Ghosh}
\date{\today}
\vfill


% Main Content



\maketitle

\begin{abstract}
    Extreme weather phenomena, such as excessive rainfaill and sea-level anomalies (SLAs), 
    make the Western Maritime Continent (WMC) vulnerable to flooding. Frequent flooding between
    November and February occurs due to the combination of high speed winds over
    the South China Sea (SCS) and SLAs. These winds are characteristic of the boreal winter monsoon
    cold surge (CS) events that occur during the northeast monsoon season. The role of precipitation 
    anomalies is evident from the recent flooding that occurred in Jalan Seaview this January, where
    rainfall higher than the monthly average of 224.2mm coincided with a high tide of 2.8m. The role of
    SLAs in such cases is undeniable, but the literature on this topic remains incomplete. The primary 
    objective of this study is to quantify the role that CS events play on `extreme tides'. We use research-quality
    water level data from tide gauges, obtained from the University of Hawaii Sea Level Centre from $1990-2013$ along
    the WMC. By defining `extreme tides' as water levels about the $99$\textsuperscript{th}-percentile, we have quantified
    the co-occurrences of CS events and `extreme tides', along with the associated increase in average water levels. Henceforth,
    we develop and test a model based on water level, wind speed, and pressure-level anomalies, to obtain a relation between CS events
    and extreme water levels. Development of this model can improve understanding of surge-induced floods, with 
    implications for disaster preparedness in Sotuh-East Asia, thus aiding government agencies in constructing 
    early warning systems and risk mitigation for vulnerable areas.
\end{abstract}

\end{document}
