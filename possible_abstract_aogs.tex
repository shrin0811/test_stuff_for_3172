\documentclass[12pt]{article}
\usepackage{a4wide}
\usepackage{amsmath, amssymb}
\usepackage{graphicx}
\usepackage{hyperref}
\hypersetup{
    colorlinks=true,
    linkcolor=blue,
    filecolor=magenta,      
    urlcolor=blue
}

\begin{document}
\title{\vspace{-2.0cm}Effect of monsoon surges on extreme tides in the Western Maritime Continent}
\author{Shrinjana Ghosh}
\date{\today}
\vfill


% Main Content



\maketitle

\begin{abstract}
    Climate change will not only increase the intensity, but also the frequency of flash floods, with the island-nation of Singapore 
    being hit by three episodes within a span of three months between October to December in the year 2024. The combined effect of high
    wind speeds over the South China Sea (SCS), internal climate variabilties (such as the El-Nino Southern Oscillation (ENSO)) and spring tides
    have been linked to coastal and flash floods (such as the ones seen on 23\textsuperscript{rd} Decemeber 1999).
    These SCS winds are characteristic of the (boreal winter) monsoon cold surge (CS) events that 
    occur over the Maritime Continent during the northeast (NE) monsoon season. The primary motivation of this study is to quantify the role that CS events
    play on extreme events. The study uses hourly tide gauge (research quality) data from the University of Hawaii Sea Level Centre from $1990-2013$ along the 
    Western Maritime Continent. Further analysis of the data has allowed us to quantify `extreme tides', the co-occurrences of CS events and `extreme tides', along with the
    average increase in tide levels associated with CS events. Henceforth, we develop and test a model based on tide level, wind speed and pressure level anomalies,
    along with other climatic variables associated with CS events to predict the daily maximum tide levels and `extreme tides'. The development of this model
    aims to improve predictability for surge-induced flash floods, with implications for disaster preparedness in South-East Asia, thus aiding
    government agencies in risk mitigation for vulnerable areas and early warning systems.
    \newline (244 words)
\end{abstract}



\begin{paragraph}
    \noindent Points to be included in the abstract according to SPS guidelines: 
\begin{itemize}
    \item Introduction and background literature review
    \begin{itemize}
        \item For this: can use the literature review already provided by Xin Rong
    \end{itemize}
    \item Short brief as to why this question is important to solve: explain the research gap 
    \item Short brief into the methodology and techniques used 
    \item How does it fit into the `bigger scheme of things?'
\end{itemize}
\end{paragraph} 
\subsection*{Sample abstract 1}
    \href{https://link.springer.com/article/10.1007/s11069-019-03596-2}{Pham et. al (2019)}
    \begin{paragraph}
        \noindent With sea levels projected to rise as a result of climate change, it is imperative to understand
not only long-term average trends, but also the spatial and temporal patterns of extreme sea
level. In this study, we use a comprehensive set of 30 tide gauges spanning 1954–2014 to
characterize the spatial and temporal variations of extreme sea level around the low-lying and densely populated margins of the South China Sea. We also explore the long-term evo-
lution of extreme sea level by applying a dynamic linear model for the generalized extreme value distribution (DLM-GEV), which can be used for assessing the changes in extreme
sea levels with time. Our results show that the sea-level maxima distributions range from
~90 to 400 cm and occur seasonally across the South China Sea. In general, the sea-level
maxima at northern tide gauges are approximately $25-30\%$ higher than those in the south and are highest in summer as tropical cyclone-induced surges dominate the northern sig-
nal. In contrast, the smaller signal in the south is dominated by monsoonal winds in the winter. The trends of extreme high percentiles of sea-level values are broadly consistent
with the changes in mean sea level. The DLM-GEV model characterizes the interannual variability of extreme sea level, and hence, the 50-year return levels at most tide gauges.
We find small but statistically signifcant correlations between extreme sea level and both
the Pacifc Decadal Oscillation and El Niño/Southern Oscillation. Our study provides new
insight into the dynamic relationships between extreme sea level, mean sea level and the
tidal cycle in the South China Sea, which can contribute to preparing for coastal risks at
multi-decadal timescales.
\end{paragraph}

\subsection*{Sample Abstract 2}
    \href{https://dspace.mit.edu/handle/1721.1/106471}{Tkalich et. al(2013)}
    \begin{paragraph}
        \noindent Among the semi-enclosed basins of the world ocean, the South China Sea (SCS) is unique in its configuration
        as it lies under the main southwest-northeast pathway of the seasonal monsoons. The northeast (NE) monsoon
        (November-February) and southwest monsoon (June-August) dominate the large scale sea level dynamics of
        the SCS. Sunda Shelf at the southwest part of SCS tends to amplify Sea Level Anomalies (SLAs) generated
        by winds over the sea. The entire region, bounded by Gulf of Thailand on the north, Karimata Strait on the
        south, east cost of Peninsular Malaysia on the west, and break of Sunda Shelf on the east, could experience
        positive or negative SLAs depending on the wind direction and speed. Strong sea level surges during NE
        monsoon, if coincide with spring tide, usually lead to coastal floods in the region. To understand the
        phenomena, we analyzed the wind-driven sea level anomalies focusing on Singapore Strait (SS), laying at the
        most southwest point of the region. An analysis of Tanjong Pagar (TG) tide gauge data in the SS, as well as
        satellite altimetry and reanalyzed wind in the region, reveal that the wind over central part of SCS is arguably
        the most important factor determining the observed variability of SLAs at hourly to monthly scales.
        Climatological SLAs in SS are found to be positive, and of the order of 30 cm during NE monsoon, but
        negative, and of the order of 20 cm during SW monsoon. The largest anomalies are associated with
        intensified winds during NE monsoon, with historical highs exceeding 50 cm. At the hourly and daily timescales, SLA magnitude is correlated with the NE wind speed over central part of SCS with an average time lag
        of 36 to 42 hours. An exact solution is derived by approximating the elongated SCS shape with onedimensional two-step channel. The solution is utilized to fit empirical function connecting SLAs in SS with
        the wind speeds over central part of SCS. Due to delay of sea level anomaly in SS with respect to the remote
        source at SCS, the simplified solutions could be used for storm surge forecast, with a lead time exceeding one
        day. 
    \end{paragraph}
\end{document}
